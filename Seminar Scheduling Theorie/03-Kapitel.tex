\chapter{Zusammenfassung}

Der Schwerpunkt dieser Arbeit war Constraint Programmierung und ihre Anwendung in Scheduling-Theorie.

Im ersten Teil wurden wichtige Begriffe und Algorithmen dieser Programmierparadigma erklärt. Außerdem es wurde in die Vorteile des Constraint Programmierung eingegangen und ein kleines Vergleich zwischen ihr und ganzzahliger Optimierung gemacht. Das mach die Arbeit zu einer kurzen Einführung in Constraint Programmierung für Anfänger. 

Im zweiten Teil wurden einige Beispiele aus der Scheduling Theorie betrachtet und wie sie mit Hilfe von Constraint Programmierung formuliert und gelöst werden können. Es wurde mit einem einfacheren Beispiel der Minimierung der Gesamtverspätung angefangen, um zu zeigen wie einfach ein Problem in CP formuliert werden kann. Und das Beispiel mit Timetabling am Ende zeigt noch mal, das Benutzung von Contraint Programmierung anstatt ganzzahlige Optimierung zu signifikanten Verbesserung der Laufzeit führen kann.

Es existiert aber weitere interessante Ansätze der CP im Bereich der Scheduling-Theorie, die in der Arbeit nicht betrachtet wurden. In dieser Richtung kann die Arbeit erweitert und vervollständigt werden. 