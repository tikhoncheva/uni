\chapter*{ Einleitung}
\label{sec:Einletung}

\FloatBarrier 
Die Constraint Logische Programmierung (engl. Constraint Logic Programming) oder einfach Constraint Programmierung (kurz $CP$) ist ein Programmierparadigma, das unbekannte Variablen und Beziehungen zwischen ihnen durch Nebenbedingungen (Constraints) beschreibt. 

Es kam ursprünglich aus dem Bereich der künstlichen Intelligenz und gilt als eine Erweiterung der Ideen der logischen Programmierung.

Der Constraint-Programmierung liegen zwei Hauptprinzipien zugrunde \citep[vgl][]{CBScheduling} : Deduktion der zusätzlichen Nebenbedingungen aus den vorhandenen durch logische Folgerungen und Anwendung der Suchalgorithmen zum Untersuchen des Lösungsraums. 

Das Betrachten von Problemen bezüglich der logischen Beziehungen zwischen ihren Objekten erlaubt oft eine einfachere und natürliche Formulierung des Modells für diese Probleme. Kombiniert mit der Benutzung effektiver Suchalgorithmen macht dies die Anwendung von Constraint-Programmierung sehr attraktiv für viele kombinatorische Probleme.

In dieser Arbeit betrachten wir den Ansatz der Constraint-Programmierung in der Scheduling Theorie.

Die Arbeit besteht aus $3$ Kapiteln. Im ersten machen wir den Leser mit den Grundideen und Algorithmen der Constraint-Programmierung bekannt. Wir vergleichen auch die Unterschiede zwischen der Benutzung von Mixed-Integer-Programmierung und Constraint Programmierung zur Lösung von Scheduling-Problemen. Im zweiten Kapitel betrachten wir die Anwendungsbeispiele von CP auf konkrete Scheduling-Probleme.

Diese Ausarbeitung orientiert sich hauptsächlich an den Definitionen und der
Notation aus \cite{CPforScheduling}. Die zusätzliche Quellen werden an jeder Stelle explizit kenntlich gemacht.

