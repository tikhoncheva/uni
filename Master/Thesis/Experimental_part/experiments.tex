% ---------------------------------------          Preamble
\documentclass[
	fontsize=12pt,
	paper=a4,
	twoside=false,
	numbers=noenddot,
	plainheadsepline,
	toc=listof,
	toc=bibliography
]{scrartcl}

\usepackage[english]{babel} 

\usepackage{amssymb}
\usepackage{amsmath}
\usepackage{array}

\usepackage{placeins}
\usepackage{float}

\usepackage{graphicx}
\restylefloat{figure}
\usepackage{caption}
\usepackage{subcaption}
%\usepackage{subfigure} 
\usepackage{tikz}

%\usepackage{pdfpages} % insert images saved as pdf

% pseudo algorithms
\usepackage[ruled,vlined]{algorithm2e}

% lscape.sty Produce landscape pages in a (mainly) portrait document.
\usepackage{lscape}

\usepackage{hyperref}

\setlength{\parindent}{0pt}

\usepackage[sort, numbers]{natbib}
% ---------------------------------------          New commands
%\newcommand{\argmin}{\operatornamewithlimits{argmin}}
\def\argmax{\mathop{\rm argmax}}						% argmax
\def\argmax{\mathop{\rm argmin}}						% argmin
\def\median{\mathop{\rm median}} 						% median
\def\dist{\mathop{\rm dist}} 						    % dist

\makeatletter
\providecommand\phantomcaption{\caption@refstepcounter\@captype}
\makeatother


\usepackage{array}
\newcolumntype{L}[1]{>{\raggedright\let\newline\\\arraybackslash\hspace{0pt}}m{#1}}
\newcolumntype{C}[1]{>{\centering\let\newline\\\arraybackslash\hspace{0pt}}m{#1}}
\newcolumntype{R}[1]{>{\raggedleft\let\newline\\\arraybackslash\hspace{0pt}}m{#1}}


\usepackage{xcolor} 
\newcommand\ToDo[1]{\textcolor{red}{#1}} 

% ------------------------------------------------------------------------------------------------------------
% ------------------------------------------------------------------------------------------------------------
% ------------------------------------------------------------------------------------------------------------

\begin{document}

\pagestyle{plain}
\pagenumbering{arabic}

% ------------------------------------------------------------------------------------------------------------
% ---------------------------------------          Title
\title{Graph Matching Framework}
\author{Ekaterina Tikhoncheva}
\date{} 

\maketitle 



% ------------------------------------------------------------------------------------------------------------
% ---------------------------------------        Experimantal Evaluation
\section{Experimental Evaluation}

In this chapter we present the evaluation results of the proposed algorithm on some synthetic data and on some real images.

\subsection{Synthetic Point set Matching}

For the first test we adopted a commonly used approach of evaluation Graph Matching algorithms on the synthetic generated set of nodes (see \cite{Cho2014_Haystack}, \cite{Cho2010_RRWM}, \cite{Leordeanu2009_IPFP}). 

For this propose one generates first $n_1$ normal distributed points $V_1\subset\mathbb{R}^2$ with zero mean and standard deviation $1$. The second set $V_2$ is created from the first one by adding noise $\mathcal{N}(0,\sigma^2)$ to the positions of points in $V_1$ and $m$ additional normal distributed points with $\mathcal{N}(0,1)$.  That means, that the set $V_2$ consists of $n_2=n_1+\bar{n}$ nodes, where $n_1$ points are inliers and $\bar{n}$ points are outliers. The task is to find the correspondences between points in two sets.

In this test we follow the setup in \cite{Cho2014_Haystack} and compare our approach with following state of the art methods: \emph{MPM}~\cite{Cho2014_Haystack}, \emph{RRWM}~\cite{Cho2010_RRWM}, \emph{SM}~\cite{Leordeanu2005}, \emph{IPFP}~\cite{Leordeanu2009_IPFP}. Because of those algorithms work with the full affinity matrix of the Graph Matching Problem, whose size is equal to $n_1n_2\times n_1n_2$, it is time and memory consuming to perform tests for graphs with more than $200$ nodes each. Our algorithms, however, was created to work with graphs bigger than that. To be able to perform the comparison, we fixed the number $n_1$ of points in the first set to $100$ and vary the number of outliers $\bar{n}$ in the second set from $0$ to $50$. The discretization of the continuous solution is performed in all cases using greedy assignment from \cite{Leordeanu2005}.

For the first test we set number of outliers $\bar{n}$ to zero and vary only the deformation noise $\sigma^2$. We call this test \emph{deformation test}. It's results are shown at the Fig.~\ref{fig:def_test1} and \ref{fig:def_test2}.

In the second test, $outlier test$, we fix deformation noise $\sigma^2= 0.03$ and compare the behavior of the algorithms in case of increasing number of outliers $\bar{n}$ (see Fig.~\ref{fig:outlier_test1} and \ref{fig:outlier_test2}). 

\begin{figure}[h] 
	\begin{subfigure}[b]{0.3\textwidth}
		\centering
		\includegraphics[scale=0.25]{"fig_ver2108/syntheticPointSets/deformation_test/accuracy_greedy"} 
%		\caption{} 
	\end{subfigure}%% 
	\begin{subfigure}[b]{0.3\textwidth}
		\centering
		\includegraphics[scale=0.25]{"fig_ver2108/syntheticPointSets/deformation_test/score_greedy"} 
%		\caption{} 
	\end{subfigure} 
	\begin{subfigure}[b]{0.3\textwidth}
		\centering
		\includegraphics[scale=0.25]{"fig_ver2108/syntheticPointSets/deformation_test/time_greedy"} 
%		\caption{} 
	\end{subfigure} 	
	\caption{ Deformation test: $n_1=100$, $n_2=100$, $\sigma^2\in[0, 0.2]$}
	\label{fig:def_test1}
\end{figure}

\FloatBarrier	

\begin{figure}[h] 
	\begin{subfigure}[b]{0.3\textwidth}
		\centering
		\includegraphics[scale=0.25]{"fig_ver2108/syntheticPointSets/deformation_test/accuracy_avg10tests"} 
%		\caption{} 
	\end{subfigure}%% 
	\begin{subfigure}[b]{0.3\textwidth}
		\centering
		\includegraphics[scale=0.25]{"fig_ver2108/syntheticPointSets/deformation_test/score_avg10tests"} 
%		\caption{} 
	\end{subfigure} 
	\begin{subfigure}[b]{0.3\textwidth}
		\centering
		\includegraphics[scale=0.25]{"fig_ver2108/syntheticPointSets/deformation_test/time_avg10tests"} 
%		\caption{} 
	\end{subfigure} 	
	\caption{ Average of $10$ deformation tests: $n_1=100$, $n_2=100$, $\sigma^2\in[0, 0.2]$}
	\label{fig:def_test2}	
\end{figure}

\FloatBarrier	

\begin{figure}[h] 
	\begin{subfigure}[b]{0.3\textwidth}
		\centering
		\includegraphics[scale=0.25]{"fig_ver2108/syntheticPointSets/outliertest_n50/accuracy_greedy"} 
		%		\caption{} 
	\end{subfigure}%% 
	\begin{subfigure}[b]{0.3\textwidth}
		\centering
		\includegraphics[scale=0.25]{"fig_ver2108/syntheticPointSets/outliertest_n50/score_greedy"} 
		%		\caption{} 
	\end{subfigure} 
	\begin{subfigure}[b]{0.3\textwidth}
		\centering
		\includegraphics[scale=0.25]{"fig_ver2108/syntheticPointSets/outliertest_n50/time_greedy"} 
		%		\caption{} 
	\end{subfigure} 	
	\caption{ Outliers test:$n_1=100$, $\bar{n}\in[0,50]$, $\sigma^2=0.03$}
	\label{fig:outlier_test1}
\end{figure}

\FloatBarrier	

\begin{figure}[h] 
	\begin{subfigure}[b]{0.3\textwidth}
		\centering
		\includegraphics[scale=0.25]{"fig_ver2108/syntheticPointSets/outliertest_n50/accuracy_avg10tests"} 
		%		\caption{} 
	\end{subfigure}%% 
	\begin{subfigure}[b]{0.3\textwidth}
		\centering
		\includegraphics[scale=0.25]{"fig_ver2108/syntheticPointSets/outliertest_n50/score_avg10tests"} 
		%		\caption{} 
	\end{subfigure} 
	\begin{subfigure}[b]{0.3\textwidth}
		\centering
		\includegraphics[scale=0.25]{"fig_ver2108/syntheticPointSets/outliertest_n50/time_avg10tests"} 
		%		\caption{} 
	\end{subfigure} 	
	\caption{ Average of $10$ outlier tests:$n_1=100$, $\bar{n}\in[0,50]$, $\sigma^2=0.03$}
	\label{fig:outlier_test2}
\end{figure}

\FloatBarrier

\subsection{Image Affine Transformation}


\subsection{Real Images}
% ------------------------------------------------------------------------------------------------------------
% ---------------------------------------        Bibliography
\bibliographystyle{abbrv}
\bibliography{bibliography}
	
\end{document}