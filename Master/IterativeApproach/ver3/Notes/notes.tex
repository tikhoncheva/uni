\newcommand{\argmin}{\operatornamewithlimits{argmin}}
% ---------------------------------------          Preamble                   -------------------------------------------------------- 
\documentclass[
	fontsize=12pt,
	paper=a4,
	twoside=false,
	numbers=noenddot,
	plainheadsepline,
	toc=listof,
	toc=bibliography
]{scrartcl}

\usepackage[english]{babel} 

\usepackage{amssymb}
\usepackage{amsmath}

\usepackage{placeins}
\usepackage{float}

\usepackage{graphicx}
\restylefloat{figure}
\usepackage{subfigure} 

\usepackage{array}

\usepackage{hyperref}

\setlength{\parindent}{0pt}

\usepackage[sort, numbers]{natbib}

\begin{document}

\pagestyle{plain}
\pagenumbering{arabic}

% ---------------------------------------          Title                --------------------------------------------------
\title{Graph Matching Framework}
\author{Ekaterina Tikhoncheva}
\date{} 

\maketitle 

% ---------------------------------------          Introduction         --------------------------------------------------
This notes are a short description of graph matching model applied for finding feature correspondences between two images.

% Inhaltsverzeichnis erzeugen
\tableofcontents
\newpage

% ---------------------------------------        Problem Statement      --------------------------------------------------
\section{Problem statement}
Consider two undirected weighted graphs $G^I = (V^I, E^I, A^I)$ and $G^J = (V^J, E^J, A^J)$, where $V$, $E$, $A$ denote set of nodes,
set of edges and set of node attributes respectively. We assume situation, where $|V^I|=n_1$, $|V^J|=n_2$ and $n_1$ is not necessary equal to $n_2$.

The aim of graph matching is to find a subset of possible node correspondences, which maximizes the similarity value between two graphs. Such subset can be represented by a binary vector $x\in \{0,1\}^{n_1n_2}$, where $x_{(j-1)n_1+i}=1$, if node $v_i\in V^I$ is matched to node $u_j\in V^J$, and $x_{(j-1)n_1+i}=0$ otherwise. For simplicity we will write further $x_{ij}$ instead of $x_{(j-1)n_1+i}$.

To measure similarity between graphs we define two similarity functions: \emph{nodes similarity function} (first-order similarity) $s_V(v_i, u_j),\ v_i\in V^I, u_j\in V^J$ and \emph{edge similarity function} (second-order similarity) $s_E(e_{ii'}, e_{jj'}),\ e_{ii'}\in E^I, e_{jj'}\in E^J$. Both functions can be combined in one \emph{similarity matrix $S\in\mathbb{R}^{n_1n_2\times n_1n_2}$}, whose diagonal elements are $s_V(v_i, u_j)$ and non-diagonal elements are $s_E(e_{ii'}, e_{jj'})$.


Using this notation one can formulate \emph{one-to-one graph matching problem} as an quadratic optimization problem (\cite{Cho2014_Haystack}, \cite{Cho2012_ProgressiveGM}, \cite{Cho2010_RRWM}): 
\begin{alignat}{2}
    &     && \argmin_x{x^TSx}                           \label{QIP::1}\\
    & \text{s.t. } &&  x\in \{0,1\}^{n_1n_2}            \label{QIP::2}\\
    &             &&  \sum_{i=1\dots n_1} x_{ij} = 1    \label{QIP::3}\\
    &             &&  \sum_{j=1\dots n_2} x_{ij} = 1    \label{QIP::4}
 \end{alignat}
 
The maximum number of possible matches is equal to $\min(n_1, n_2)$. That means, in case when $n_1\not = n_2$, only one of the conditions (\ref{QIP::3}) or (\ref{QIP::4}) will be fulfilled.

Quadratic Optimization Problem is known to be \emph{NP}-hard \cite{Sahni1974}. This limits greatly the size of a graph, for which a exact solution can be calculated in reasonable time. Due to this there is a number of algorithms () that solve graph matching problem inexact.

A standard approach to solve formulated problem approximately is to relax the integrality constrains: $x\in [0,1]^{n_1n_2}$ instead of  $x\in \{0,1\}^{n_1n_2}$. To return back to discrete solution one can apply Greedy Matching or Hungarian Algorithm \cite{Kuhn1955} on obtained continues solution.

Unfortunately, most of the algorithms are two following problems:
\begin{enumerate}
\item they are still limited in size of permissible graphs. Experiments in most of the papers consider graphs with up to $100$ nodes.
\item possible presence of outliers can reduce the accuracy of matching algorithm (\cite{Suh_CVPR2015}).
\end{enumerate}  

Our main aim was to develop a framework, which would allow an existing graph matching algorithm to cope with both problems.
% -----------------------------------------------------------------------------------------------------------------------



% ---------------------------------------        Approach      --------------------------------------------------
\section{Approach}

The main idea of our approach is to perform graph matching on several stages. Given initial graphs $G^I$ and $G^J$ we create for each of them a coarse representative graph $$.


% -----------------------------------------------------------------------------------------------------------------------



% ---------------------------------------        LLG Construction      --------------------------------------------------
\subsection{Lower Level Graph Construction}

% -----------------------------------------------------------------------------------------------------------------------


% ---------------------------------------        HLG Construction      --------------------------------------------------
\subsection{Higher Level Graph Construction}
% -----------------------------------------------------------------------------------------------------------------------


% ---------------------------------------        Matcing algorithm      --------------------------------------------------
\subsection{Matching Algorithm}
% -----------------------------------------------------------------------------------------------------------------------


% ---------------------------------------        Level connection      --------------------------------------------------
\subsection{Connection between two levels}
% -----------------------------------------------------------------------------------------------------------------------


%\bibliographystyle{plain}
\bibliography{bibliography}
	
\end{document}